\subsection{Stochastic Nonlinear Model}
\label{sec:Stoc_Model}
\subsubsection{Dynamics of the system - Point 1}
It is desired to represent the system as mathematical model in the following form.
\begin{equation}
    \dot{x}(t)=f(x(t),u(t),d(t),p)\;,\;d(t)=d_k(t)\,\text{for}\,t_k\leq
    t\leq t_{k+1}
\end{equation}
The definition of $d_k(t)$ is related to the assumption that the disturbances $F_3$ and $F_4$ are stochastic variables but piecewise constant. Normally, a stochastic variable is modelled as Brownian motion, however this is not well defined in mathematical point of view. Therefore it is modelled as piecewise constant meaning that in a given time interval ($t_k$ to $t_{k+1}$) the disturbance is constant. By this simplifying assumption, the general model described in \cref{eq:States} in \cref{sec:Dynamics} can be rewritten into the following. \begin{equation}
    \begin{gathered}
        \dot{x}_1(t)=\rho\left(\gamma_1\,u_1(t)+a_3\sqrt{2\,g\,\frac{x_3(t)}{\rho\,A_3}}-a_1\sqrt{2\,g\,\frac{x_1(t)}{\rho\,A_1}}\right)\\
        \dot{x}_2(t)=\rho\left(\gamma_2\,u_2(t)+a_4\sqrt{2\,g\,\frac{x_4(t)}{\rho\,A_4}}-a_2\sqrt{2\,g\,\frac{x_2(t)}{\rho\,A_2}}\right)\\
        \dot{x}_3(t)=\rho\left((1-\gamma_2)\,u_2(t)+d_{k1}(t)-a_3\sqrt{2\,g\,\frac{x_3(t)}{\rho\,A_3}}\right)\;,\,d_{k1}(t)\sim N(0,R_{QQ}(p))\\
        \dot{x}_4(t)=\rho\left((1-\gamma_1)\,u_1(t)+d_{k2}(t)-a_4\sqrt{2\,g\,\frac{x_4(t)}{\rho\,A_4}}\right)\;,\,d_{k2}(t)\sim N(0,R_{QQ}(p))\\
    \end{gathered}
    \label{eq:stoc_states}
\end{equation}

\subsubsection{Sensors - Point 2}
It is desired to model the measurements which always will be affected by some measurement noise, leading the output equation to take the following form.
\begin{equation}
    y(t)=g(x(t),p)+v(t)\;,\,v(t)\sim N(0,R_{vv}(p))
\end{equation}
The noise is modelled as a normal distribution with 0 mean and the covariance $R_{vv}(p)$. The mathematical expression for each sensor is seen below.
\begin{equation}
    \begin{gathered}
        y_1(t)=\frac{x_1(t)}{\rho\,A_1}+v_1(t)\\
        y_2(t)=\frac{x_2(t)}{\rho\,A_2}+v_2(t)
    \end{gathered}
\end{equation}

\subsubsection{Outputs - Point 3}
The mathematical model for the outputs is not affected by the noise and can therefore be modellled as the deterministic case.
\begin{equation}
    \begin{gathered}
        h_1(t)=\frac{x_1(t)}{\rho\,A_1}\\
        h_2(t)=\frac{x_2(t)}{\rho\,A_2}\\
        h_3(t)=\frac{x_3(t)}{\rho\,A_3}\\
        h_4(t)=\frac{x_4(t)}{\rho\,A_4}
    \end{gathered}
\end{equation}