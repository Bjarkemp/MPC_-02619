\subsection{Comparison of MPC's}
An advantage of the classical controllers is that they are relatively easy to implement and it required a low computational power while the system is online. However, they do not have the ability to "predict" ahead of time and therefore adjust prior to changes. They have no knowledge of unknown inputs, such as disturbances, which is used together with classic MPC (in case of a Kalman filter). \\
The main advantage of MPC is the ability to predict future reference changes. This allows the system to begin the control action in advance, which reduces settling time. Also the ability to design the MPC with knowledge about the input constraints (which can be used to ensure saturation of actuators do not occur), the outputs constraints and the rate of movement in the input.  However the cons of these MPC is they are relatively difficult to implement and require more processing power (the more constraint, the more demanding).\\
The MPC can be implemented with an economic perspective by using the Economic MPC, which allows the design to take into account the cost of the performance instead of evaluating the control only with respect to performance. This case can roughly be described as a special case of the soft output constrained MPC.
\subsection{Conclusion}
In this report the advantage of different MPC's have been showed. The MPC works well with stochastic systems, where there is unknown inputs. Eventhough the MPC is build on a linear model, the performance is well for non-linear systems. The advantage of constraints on the MPC has been shown useful (especially in relation to real life implementation). 
