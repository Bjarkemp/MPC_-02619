\subsection{Continious Time State Space Models}
\subsubsection{Deterministic Model}
\label{sec:SS_det}
The non-linear model derived in \cref{sec:Det_Model} can be transformed into a state space model through linearization. A Linear Time Invariant (\textit{LTI}) system is only valid close to the linerization point. The system in state space form is given by the below equation.
\begin{equation}
    \begin{gathered}
        \dot{x}(t)=A\,X(t)+B\,U(t)+G\,D(t)\\
        y(t)=C\,X(t)\\
        z(t)=C_z\,X(t)\\
        X(t)=x(t)-x_{ss} \quad U(t)=u(t)-u_{ss} \quad D(t)=d(t)-d_{ss}
    \end{gathered}
\end{equation}
It should be noticed that the measurements ($y$) and the controlled variable ($z$) is the same for the given system. One should notice the use of the states and inputs is deviation variables, meaning the steady state (denoted ss) is subtracted from the respective state.\\
The matrices is determined by differentiation the respective equation with respect to the given state, input and disturbance. The derived algebraic expression for each matrix is seen below.
\begin{equation}
    \begin{gathered}
        A=\begin{bmatrix}
        -\frac{\sqrt{2}\,a_1\,g}{2\,A_1\,\sqrt{\frac{g\,m_1}{A_1\,\rho}}} & 0 & \frac{\sqrt{2}\,a_3\,g}{2\,A_3\,\sqrt{\frac{g\,m_3}{A_3\,\rho}}} & 0\\
        0 & -\frac{\sqrt{2}\,a_2\,g}{2\,A_2\,\sqrt{\frac{g\,m_2}{A_2\,\rho}}} & 0 & \frac{\sqrt{2}\,a_4\,g}{2\,A_4\,\sqrt{\frac{g\,m_4}{A_4\,\rho}}}\\
        0 & 0 & -\frac{\sqrt{2}\,a_3\,g}{2\,A_3\,\sqrt{\frac{g\,m_3}{A_3\,\rho}}} & 0\\
        0 & 0 & 0 & -\frac{\sqrt{2}\,a_4\,g}{2\,A_4\,\sqrt{\frac{g\,m_4}{A_4\,\rho}}}
        \end{bmatrix} \qquad
        C=\begin{bmatrix}
            \frac{1}{A_1\,\rho} & 0 & 0 & 0\\
            0 & \frac{1}{A_2\,\rho} & 0 & 0
        \end{bmatrix} \\
        B=\begin{bmatrix}
        \gamma_1\,\rho & 0\\
        0 & \gamma_2\,\rho\\
        0 & (1-\gamma_2)\,\rho\\
        (1-\gamma_1)\,\rho & 0
        \end{bmatrix} \qquad
        G=\begin{bmatrix}
            0 & 0\\
            0 & 0\\
            \rho & 0\\
            0 & \rho
        \end{bmatrix} 
    \end{gathered}
\end{equation}
By analyzing the $A$ matrix, the correlation between the tanks is clearly visible. Tank 1 and 2 is depending on the water level in tank 3 and 4 respectively. The $C$ matrix indicates that is only possible to measure the water level in tank 1 and 2 (notice, that $C_z$ is not shown, since $C=C_z$). The $B$ matrix indicates that it is possible to affect the water in all tanks. The $G$ matrix indicates that the disturbance is only affecting tank 3 and 4 directly.\\
As mentioned, the state space model is only valid close to the linearization point, which is the steady state point. In this project, the steady state values is given by \cref{eq:ss_values}. The corresponding state space model at the given linearization point is seen in \cref{eq:SS_ss}
\begin{equation}
    x_{ss}=\begin{bmatrix}
    3.4778 \\ 4.3254 \\ 1.5401 \\ 1.8188
    \end{bmatrix}\cdot10^4\,[g] \quad
    y_{ss}=\begin{bmatrix}
    91.49 \\ 113.79
    \end{bmatrix}\,[cm] \quad
    u_{ss}=\begin{bmatrix}
    300 \\ 300
    \end{bmatrix}\,[cm^3/s] \quad
    d_{ss}=\begin{bmatrix}
    250 \\ 250
    \end{bmatrix}\,[cm^3/s] \quad
    \label{eq:ss_values}
\end{equation}
\begin{equation}
    \begin{gathered}
        A = \begin{bmatrix}
            -0.0075 & 0 & 0.0112 & 0\\
            0 & -0.0067 & 0 & 0.0103\\
            0 & 0 & -0.0112 & 0\\
            0 & 0 & 0 & -0.0103
        \end{bmatrix} \qquad 
        C=\begin{bmatrix}
            0.0026 & 0 & 0 & 0\\
            0 & 0.0026 & 0 & 0
        \end{bmatrix} \\
        B=\begin{bmatrix}
            0.5800 & 0\\
            0 & 0.6800\\
            0 & 0.3200\\
            0.4200 & 0
        \end{bmatrix} \qquad
        G=\begin{bmatrix}
            0 & 0\\
            0 & 0\\
            1 & 0\\
            0 & 1
        \end{bmatrix} 
    \end{gathered}
    \label{eq:SS_ss}
\end{equation}
\subsubsection{Stochastic Model (piecewise Constant)}
The stochastic model defined in \cref{sec:Stoc_Model} was determined under the assumption that the disturbances is piecewise constant. This implies that the model in each time interval is deterministic, and therefore the linearized stochastic model is almost identical to the stochastic model determine the previous paragraph.
\begin{equation}
    \begin{gathered}
        \dot{x}(t)=A\,X(t)+B\,U(t)+G\,D(t)\\
        y(t)=C\,X(t)+I\,v(t)\\
        z(t)=C_z\,X(t)\\
        X(t)=x(t)-x_{ss} \quad U(t)=u(t)-u_{ss} \quad D(t)=d(t)-d_{ss} \quad v\sim N(0,R_{vv}(p)
    \end{gathered}
\end{equation}
The only difference in this state space representation is the noise contribution to the output, is random, however this do not affect the system matrices. 
