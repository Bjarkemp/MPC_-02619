\section{Closed-Loop Simulations}
In this section the MPC's determined in \cref{sec:uncon_MPC}, \ref{sec:in_con_MPC} and \ref{sec:in_out_con_MPC} will be simulated on both the linear models and the non-linear models.\\
The reference signal is modelled as shown in \cref{tab:CL_sim_ref}. The system is initialized from steady state, where $u=\begin{bmatrix} 300 & 300\end{bmatrix}^T$. The disturbance is set be stochastic variable which follows a normal distribution given by: $d=N(250,\sqrt{10})$. The measurement noise is modelled as normal distributed according to $v_k=N(0,\sqrt{20})$.
\begin{table}[H]. 
    \centering
    \begin{tabular}{|ccc|} \hline
         Time [min] & $\Delta r_1$ [m] & $\Delta r_2$ [m] \\ \hline
         0 & 0 & 0\\
         10 & 10 & 10\\
         20 & 20 & -10 \\\hline
    \end{tabular}
    \caption{Reference signal}
    \label{tab:CL_sim_ref}
\end{table}
For all simulations, the overall concept is done by the MPC predicts 2*N (where $N=40$) number of inputs ahead. Then the first coming two rows in the prediction (due to the fact that we have two inputs) is saved as the input for the k\textsuperscript{th} iteration. This is used in both the linear model (as a deviation variable) and in the non-linear model (as absolute value). Then the non-linear and linear states is determined and finally the measurement is done. The measurement are then used as inputs to the Kalman filter in order to estimate the states, and filter out measurement noise. 
\subsection{Design of Unconstrained MPC - Point 1, 2 \& 3}
\label{sec:uncon_MPC}
The objective of the MPC is to be minimized, which can be described as a least square problem (for unconstrained MPC), which for a discrete time state space is given by
\begin{equation}
    \underset{x\in R^n}{f(x)}=\frac{1}{2}=\norm{Ax-b}_2^2=\frac{1}{2}\,x^TA^TAx-(A^Tb)^Tx+\frac{1}{2}\,b^Tb
\end{equation}
Where the expression can be rewritten in according to $H=A^TA$, $g=-A^Tb$ and $\rho=\frac{1}{2}\,b^Tb$. The unconstrained optimization problem is defined for the vector case as seen below
\begin{equation}
    \underset{x\in R^n}{f(x)}=\frac{1}{2}\,x^THx+g^Tx
    \label{eq:uncon_f}
\end{equation}
Where $\nabla f(x)=Hx+g=0$ and $\nabla^2f(x)=H\geq 0$ is necessary and sufficient optimality conditions. $H$ is positive definite and the solution is unique.\\
The MPC aims to find e.g. an input value, for which a predetermined function has a minimum. The minimization is determine based on weights $S$ and $Q$ (which is the tuning parameters).\\
The MPC is designed for the linear discrete time model determined in \cref{sec:Dis_LTI}. It is desired that the control input variation and the error on the output should be minimized, which yields the following least square problem.
\begin{equation}
    \text{min}\quad \phi=\frac{1}{2}\, \overset{N}{\underset{k=0}{\sum}}\norm{z(k)-r_k}_{Q_z}^2+\frac{1}{2}\, \overset{N-1}{\underset{k=0}{\sum}}\norm{\Delta u_k}_S^2
    \label{eq:MPC_uncon_lsq}
\end{equation}
Where $N$ is the prediction horizon, $z(k)$ is the output vector (which can be determine in according to \cref{eq:uncon_z}), $r(k)$ is the reference vector, $Q_z$ and $S$ is the weight matrices and $\Delta U_k=U_k-U_{k-1}$.\\
First the minimized output (1\textsuperscript{st} part of the equation) is analyzed.
\begin{equation}
    Z=\phi\,x_0+\Gamma\,U+\Gamma_d\,D
    \label{eq:uncon_z}  
\end{equation}
Notice that since the disturbance is unknown, $D$ is unknown. Hence, the matrix $\Gamma_d$ is not used in the simulation. Each element is determined as
\begin{equation}
    \begin{gathered}
        Z=\begin{bmatrix}
            z_1\\
            z_2\\
            z_3\\
            \vdots\\
            z_N
        \end{bmatrix} \quad
        \phi=\begin{bmatrix}
            CA\\
            CA^2\\
            CA^3\\
            \vdots\\
            CA^N
        \end{bmatrix} \quad
            \Gamma=\begin{bmatrix}
            H_1 & 0 & 0 & \dots & 0\\
            H_2 & H_1 & 0 & \dots & 0\\
            H_3 & H_2 & H_1 & \dots & 0\\
            \vdots & \vdots & \vdots & \ddots & \vdots\\
            H_N & H_{N-1} & H_{N-2} & \dots & H_1
        \end{bmatrix} \\
        H_N=CA^{i-1}B
    \end{gathered}
    \label{eq:uncon_design}
\end{equation}
The expression of $z(k)$ is now substituted into the first part of the least square problem (containing the output).
\begin{equation}
    \phi_z=\frac{1}{2}\norm{\phi\,x_0+\Gamma\,U-R}_{Q_z}^2
\end{equation}
It is desired to have the form seen in \cref{eq:uncon_f}, why the above expression is rewritten by using linear algebra, which yields
\begin{equation}
    \phi_z=\frac{1}{2}U^T\underbrace{(\Gamma^TQ_z\Gamma)}_HU+{\underbrace{(-\Gamma^TQ_zK)}_{g}}^TU+\underbrace{\frac{1}{2}K^TQ_zK}_\rho\qquad , \qquad K=R-\phi x_0     
\end{equation}
Now the minimized control input variation (2\textsuperscript{nd} part) is analyzed.
\begin{equation}
    \phi_{\Delta_u}=\frac{1}{2}\,U^TH_sU+(M_{u1}\,u_{-1})^TU
\end{equation}
Where each element matrix is determined as
\begin{equation*}
    U=\begin{bmatrix}
    u_1\\u_2\\u_3\\\vdots\\u_N
    \end{bmatrix}\qquad
    H_s=\begin{bmatrix}
    2S & -S & 0 & \dots & 0\\
    -S & 2S & -S & \dots & 0\\
    0 & -S & 2S & \dots & 0\\
    \vdots & \vdots & \vdots & \ddots & \vdots\\
    0 & 0 & 0 & -S & S
    \end{bmatrix}\qquad
    M_{u1}=-\begin{bmatrix}
    S \\ 0 \\ 0 \\ \vdots \\ 0
    \end{bmatrix}
\end{equation*}
Now the two optimizations objects is added together, so the complete minimization is
\begin{equation}
    \label{eq:MPC_obj}
    \underset{U}{min}\,\phi=\phi_z+\phi_{\Delta u}=\frac{1}{2}U^THU+g^TU
\end{equation}
Where
\begin{equation}
    \label{eq:M_mat_MPC}
    \begin{gathered}
        H=H_z+H_s=\Gamma^TQ_z\Gamma+H_s\\
        g=M_{x0}\,x_0+M_R\,R+M_{u1}\,u_{-1}\\
        M_{x0} = \Gamma^TQ_z\phi \quad M_R=-\Gamma^TQ_z
    \end{gathered}
\end{equation}
\\\\
The implementation in \textit{MatLab} is carried out through three functions. Prior to the simulation, the MPC design i carried out (see \cref{app:MPC_design}) where, $H$, $H_s$ and $g$ is determined. In order to do this, the function calls a seperate function (see \cref{app:MPC_Constants}) where $\phi$ and $\Gamma$ is determined. Finally, the qpsolver is called. Notice, that since the simulation is unconstrained, the only inputs are $H$, $H_s$ and $g$ (see \cref{app:Uncon_MPC})
\subsection{Static and Dynamic Kalman Filter}
First, the general static and dynamic Kalman filter will be derived. The one step prediction of the Kalman filter is given by
\begin{equation}
    \begin{gathered}
        R_{e,k}=C\,P_{k|k-1}\,C^T+R\\
        K_{fx,k}=P_{k|k-1}\,C^T\,R_{e,k}^{-1}\\
        P_{k+1|k}=A\,P_{k|k}\,A^T+G\,Q_{k|k}\,G^T-K_{fx,k}\,R_{e,k}\,K_{fx,k}^T
    \end{gathered}
\end{equation}
Where $R_e$ is the covariance of the measurement noise, $K_{fx}$ is the Kalman filter gain, $P$ is the covariance of the states and $Q_{k|k}$ can be determined by
\begin{equation}
    Q_{k|k}=Q-S\,R_{e,k}^{-1}\,R_{e,k}\,(S\,R_{e,k}^{-1})^T
\end{equation}
Q is the state noise, and is determined by \cref{eq:Q} and $S$ is the correlation between the states. In this assignment, the correlation is assumed to be 0, causing $S=0$. It is clearly then seen that $Q_{k|k}=Q$, which simplified is determined as.
\begin{equation}
    \begin{bmatrix}
        \phi_{11} & \phi_{12} \\ \phi_{21} & \phi_{22}
    \end{bmatrix}=
    \text{exp}\left(\begin{bmatrix} 
        -A_c & G_c\,G_c^T\\ 0 & A_c^T 
    \end{bmatrix}\,T_s\right)
    \qquad
    Q=\phi_{22}^T\,\phi_{12}
    \label{eq:Q}
\end{equation}
$P_{k+1|k}$ converges to the matrix $P$ (as $t\xrightarrow{}\infty$) which is the solution to the discrete algebraic Riccati equation (\textit{DARE}). The finite elements in the one step prediction can be determined as:
\begin{equation}
    \begin{gathered}
        R_{e}=C\,P\,C^T+R\\
        K_{fx}=P\,C^T\,R_{e}^{-1}\\
        P=A\,P\,A^T+G\,Q\,G^T-K_{fx}\,R_{e}\,K_{fx}^T
    \end{gathered}
\end{equation}
The general dynamic and static Kalman filter algorithm can be seen in \cref{tab:Kalman_gen}. Notice this algorithm is adjusted relating to the assumption of $S=0$, resulting in the filter gain to  become 0.\\
Since the disturbance is unknown, the system should be augmented in order to properly estimate the states. This leads the state space model of the system to take the following form. 
\begin{equation}
    \begin{gathered}
        x(k+1)=A_{aug}\,x(k)+B_{aug}\,u(k)+G_{aug}\,d(k)\\
        y(k)=C_{aug}\,x(k)+v(k)\\
    \end{gathered} 
\end{equation}
It is seen, that the unknown disturbances are stochastic variables $d(k)=N(0,\sqrt{Q})$.  The augmented matrices is determined according to \cref{eq:aug}. Notice, that also the state noise matrix $Q$ likewise is augmented.
\begin{equation}
    A_{aug}=\begin{bmatrix} A_k & E_k \\ 0 & I \end{bmatrix} \quad
    B_{aug}=\begin{bmatrix} B_k \\ 0 \end{bmatrix} \quad
    G_{aug}=\begin{bmatrix} G_k \\ 0 \end{bmatrix} \quad
    C_{aug}=\begin{bmatrix} C & 0 \end{bmatrix} \quad
    Q_{aug}=\begin{bmatrix} Q & 0 \\ 0 & I \end{bmatrix}
    \label{eq:aug}
\end{equation}
The augmented states is used in the Kalman algorithm. Notice that some of the steps in the algorithm is similar for both the static and dynamic filter. 
\begin{table}[H]
    \centering
    \begin{tabular}{c|c|c}
         & Dynamic & Static\\
        Measurement noise & $R_{e,k}=C_{aug}\,P_{k|k+1}\,C_{aug}^T+R$ & $R_{e}=C_{aug}\,P\,C_{aug}^T+R$\\
        Filter gain & $K_{fx,k}=P_{k|k+1}\,C_{aug}^T\,R_{e,k}^{-1}$ & $K_{fx}=P\,C_{aug}^T\,R_e^{-1}$\\
        Measurement update & \multicolumn{2}{c}{$y_{k|k-1}=C\,x_{k|k-1}+v_k$} \\
        Estimated measurement & \multicolumn{2}{c}{$\hat{y}_{k|k-1}=C_{aug}\,\hat{x}_{k|k-1}+v_k$}\\
        Error & \multicolumn{2}{c}{$e_k=y_k-\hat{y}_{k|k-1}$} \\
        Estimated states & $\hat{x}_{k|k}=\hat{x}_{k|k-1}+K_{fx,k}\,e_k$ & $\hat{x}_{k|k}=\hat{x}_{k|k-1}+K_{fx}\,e_k$ \\
        One step prediction & \multicolumn{2}{c}{$\hat{x}_{k+1|k}=A_{aug}\,\hat{x}_{k|k}+B_{aug}\,u_{k|k}+G_{aug}\,d_{k|k}$} \\
        Filtered state  & $P_{k|k}=P_{k|k-1}-K_{fx,k}\,R_{e,k}\,K_{fx,k}$ & $P_{\infty}=P-K_{fx}\,R_{e}\,K_{fx}$\\
        Covariance prediction step & $P_{k+1|k}=A_{aug}\,P_{k|k}\,A_{aug}^T+G_{aug}Q_{aug}G_{aug}^T$ & $P$
    \end{tabular}
    \caption{Generel Kalman filter algorithm}
    \label{tab:Kalman_gen}
\end{table}
Further more, it should be noted that the filter gain in order to minimize the noise ($K_{fw,k}$) is not shown, since it is 0, due to the fact that $S=0$ (no correlation of the state noise).
\subsubsection{Deterministic model}
For the deterministic model, the proces noise and measurement noise is 0, however the covariance of the measurement noise shall be $R>0$, in order to find a solution to the \textit{DARE}.
\begin{table}[H]
    \centering
    \begin{tabular}{c|c|cc}
         & Dynamic &\hspace{10mm} &Static\\
        Measurement noise & $R_{e,k}=C_{aug}\,P_{k|k+1}\,C_{aug}^T+R$ & & $R_{e}=R$\\
        Filter gain & $K_{fx,k}=P_{k|k+1}\,C_{aug}^T\,R_{e,k}^{-1}$ & &$K_{fx}=0$\\
        Measurement update & \multicolumn{3}{c}{$y_{k|k-1}=C\,x_{k|k-1}$} \\
        Estimated measurement & \multicolumn{3}{c}{$\hat{y}_{k|k-1}=C_{aug}\,\hat{x}_{k|k-1}$}\\
        Error & \multicolumn{3}{c}{$e_k=y_k-\hat{y}_{k|k-1}$} \\
        Estimated states & \multicolumn{3}{c}{$\hat{x}_{k|k}=\hat{x}_{k|k-1}$} \\
        One step prediction & \multicolumn{3}{c}{$\hat{x}_{k+1|k}=A_{aug}\,\hat{x}_{k|k}+B_{aug}\,u_{k|k}+G_{aug}\,d_{k|k}$} \\
        Filtered state  & $P_{k|k}=P_{k|k-1}-K_{fx,k}\,R_{e,k}\,K_{fx,k}$ & &$P_{\infty}=0$\\
        Covariance prediction step & $P_{k+1|k}=A_{aug}\,P_{k|k}\,A_{aug}$ & &$0$
    \end{tabular}
    \caption{Kalman filter algorithm deterministic system}
    \label{tab:Kalman_det}
\end{table}
\subsubsection{Stochastics model (piecewise constant)}
The algorithm for the dynamic and static Kalman filter is seen below.
\begin{table}[H]
    \centering
    \begin{tabular}{c|c|c}
         & Dynamic & Static\\
        Measurement noise & $R_{e,k}=C_{aug}\,P_{k|k+1}\,C_{aug}^T+R$ & $R_{e}=C_{aug}\,P\,C_{aug}^T+R$\\
        Filter gain & $K_{fx,k}=P_{k|k+1}\,C_{aug}^T\,R_{e,k}^{-1}$ & $K_{fx}=P\,C_{aug}^T\,R_e^{-1}$\\
        Measurement update & \multicolumn{2}{c}{$y_{k|k-1}=C\,x_{k|k-1}+v_k$} \\
        Estimated measurement & \multicolumn{2}{c}{$\hat{y}_{k|k-1}=C_{aug}\,\hat{x}_{k|k-1}+v_k$}\\
        Error & \multicolumn{2}{c}{$e_k=y_k-\hat{y}_{k|k-1}$} \\
        Estimated states & $\hat{x}_{k|k}=\hat{x}_{k|k-1}+K_{fx,k}\,e_k$ & $\hat{x}_{k|k}=\hat{x}_{k|k-1}+K_{fx}\,e_k$ \\
        One step prediction & \multicolumn{2}{c}{$\hat{x}_{k+1|k}=A_{aug}\,\hat{x}_{k|k}+B_{aug}\,u_{k|k}+G_{aug}\,d_{k|k}$} \\
        Filtered state  & $P_{k|k}=P_{k|k-1}-K_{fx,k}\,R_{e,k}\,K_{fx,k}$ & $P_{\infty}=P-K_{fx}\,R_{e}\,K_{fx}$\\
        Covariance prediction step & $P_{k+1|k}=A_{aug}\,P_{k|k}\,A_{aug}^T+G_{aug}Q_{aug}G_{aug}^T$ & $P$
    \end{tabular}
    \caption{Kalman filter algorithm stochastic system}
    \label{tab:Kalman_stoc}
\end{table}
\subsubsection{Simulation of Kalman Filter}
A simulation of the Kalman filter performance can be seen below. For the simulation, the system is initialized from the steady state where $u=\begin{bmatrix} 300 & 300 \end{bmatrix}^T$ and $d=\begin{bmatrix} 250 & 250 \end{bmatrix}^T$, with a step on the input of 10\% and the noise for the stochastic model is modelled as piecewise constant, where the covariance $Q$, the measurement noise covariance $R$ and the stochastic disturbance covariance $R_G$ has the following values (with $T_s=15[\text{sec}]$).
\begin{equation}
    Q=\begin{bmatrix}
        0.1152 & 0 & 1.0314 & 0\\
        0 & 0.0994 & 0 & 0.9653\\
        1.0314  & 0 & 12.7341 & 0\\
        0 & 0.9653 & 0 & 12.8971
    \end{bmatrix} \qquad R=0.5 \qquad R_G=5
\end{equation}
\begin{figure}[H] 
    \centering
    \includegraphics[width=1\textwidth]{Figures/Pr5.2_stoc_states.png}
    \caption{Kalman filter - Stochastic model states}
    \label{fig:Kalman_stoc_state}
\end{figure}
In the above figure, the states (mass in each tank) is shown. Notice, that the states is not affected directly by the measurement noise and therefore it is not shown here. It is clearly seen, that the estimate of tank 3 and 4 is poor in the beginning compared to tank 1 and 2. The reason for this, is that the system do not have a measurement of tank 3 and 4, as for tank 1 and 2 which is used to correct the estimates. The response of the dynamic and static Kalman filter is more less the same.\\
\begin{figure}[H]
    \centering
    \includegraphics[width=1\textwidth]{Figures/Pr5.2_stoc_output.png}
    \caption{Kalman filter - Stochastic model outputs}
    \label{fig:Kalman_stoc_output}
\end{figure}
In the above figure, the measured output and the estimated outputs is shown. Is is clearly seen that the measured values are corrupted by the measurements noise. The Kalman filter is able to estimate the output relatively well despite the noise.
\subsection{Input Constrained and Soft Output Constrained MPC - Closed Loop Simulation}
In this section the hard input and soft output constrained MPC is simulated on both the linear and non-linear system. For this simulation, the tuning is set as:
\begin{equation}
    \begin{gathered}
        W_z=100 \quad W_u=0 \quad W_{du}=0\\
        W_{t1}=10000 \quad W_{t2}=10000 \quad W_{s1}=10000 \quad W_{s2}=10000
    \end{gathered}
\end{equation}
The constraints is set to 
\begin{equation}
    u_{min}=0\quad u_{max}=450\quad \Delta u_{min}=-10\quad \Delta u_{max}=10 \quad Z_{min} = \begin{bmatrix} 85 \\ 105 \end{bmatrix} \quad Z_{max} = \begin{bmatrix} 110 \\ 120 \end{bmatrix}
\end{equation}
It should be noticed that the values shown in the plot is absolute values.
\begin{figure}[H]
    \centering
    \includegraphics[width=1\textwidth]{Figures/Pr10.3_InOut_con_MPC.png}
    \caption{Input constrained output constrained MPC - Simulation on linear and non-linear model}
    %\label{fig:Kalman_stoc_state_step}
\end{figure}
By first analyzing the output in tank 1, it is seen that at $T=20\,[min]$ the reference changes. The system is both limited by the input which is a hard constraint, and therefore the system is not able to reach desired output. For tank 2, the input constraints are far from the operation, so these will not limit the system. The output constraint is clearly seen from $T=10-20\,[min]$ where the output is limited. The idea of a soft constraint is clearly seen since the controller allows system to trespass the value (which is caused by the present of noise). At $T=20\,[min]$ the reference changes again, and here the lower output constraint limits the output. Once again, the presence of noise is causing the output to become lower than the limit in some small periods. \\
Due to the structure of the system, the output constraint on e.g. tank 2 can also affect tank 1. If the water level reached the constraint, this will limit the flow input. But since the flow from the pump affects both tanks, the dynamics in tank 1 can be affected.