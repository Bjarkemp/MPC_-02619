\subsection{Design of Unconstrained MPC - Point 1, 2 \& 3}
\label{sec:uncon_MPC}
The objective of the MPC is to be minimized, which can be described as a least square problem (for unconstrained MPC), which for a discrete time state space is given by
\begin{equation}
    \underset{x\in R^n}{f(x)}=\frac{1}{2}=\norm{Ax-b}_2^2=\frac{1}{2}\,x^TA^TAx-(A^Tb)^Tx+\frac{1}{2}\,b^Tb
\end{equation}
Where the expression can be rewritten in according to $H=A^TA$, $g=-A^Tb$ and $\rho=\frac{1}{2}\,b^Tb$. The unconstrained optimization problem is defined for the vector case as seen below
\begin{equation}
    \underset{x\in R^n}{f(x)}=\frac{1}{2}\,x^THx+g^Tx
    \label{eq:uncon_f}
\end{equation}
Where $\nabla f(x)=Hx+g=0$ and $\nabla^2f(x)=H\geq 0$ is necessary and sufficient optimality conditions. $H$ is positive definite and the solution is unique.\\
The MPC aims to find e.g. an input value, for which a predetermined function has a minimum. The minimization is determine based on weights $S$ and $Q$ (which is the tuning parameters).\\
The MPC is designed for the linear discrete time model determined in \cref{sec:Dis_LTI}. It is desired that the control input variation and the error on the output should be minimized, which yields the following least square problem.
\begin{equation}
    \text{min}\quad \phi=\frac{1}{2}\, \overset{N}{\underset{k=0}{\sum}}\norm{z(k)-r_k}_{Q_z}^2+\frac{1}{2}\, \overset{N-1}{\underset{k=0}{\sum}}\norm{\Delta u_k}_S^2
    \label{eq:MPC_uncon_lsq}
\end{equation}
Where $N$ is the prediction horizon, $z(k)$ is the output vector (which can be determine in according to \cref{eq:uncon_z}), $r(k)$ is the reference vector, $Q_z$ and $S$ is the weight matrices and $\Delta U_k=U_k-U_{k-1}$.\\
First the minimized output (1\textsuperscript{st} part of the equation) is analyzed.
\begin{equation}
    Z=\phi\,x_0+\Gamma\,U+\Gamma_d\,D
    \label{eq:uncon_z}  
\end{equation}
Notice that since the disturbance is unknown, $D$ is unknown. Hence, the matrix $\Gamma_d$ is not used in the simulation. Each element is determined as
\begin{equation}
    \begin{gathered}
        Z=\begin{bmatrix}
            z_1\\
            z_2\\
            z_3\\
            \vdots\\
            z_N
        \end{bmatrix} \quad
        \phi=\begin{bmatrix}
            CA\\
            CA^2\\
            CA^3\\
            \vdots\\
            CA^N
        \end{bmatrix} \quad
            \Gamma=\begin{bmatrix}
            H_1 & 0 & 0 & \dots & 0\\
            H_2 & H_1 & 0 & \dots & 0\\
            H_3 & H_2 & H_1 & \dots & 0\\
            \vdots & \vdots & \vdots & \ddots & \vdots\\
            H_N & H_{N-1} & H_{N-2} & \dots & H_1
        \end{bmatrix} \\
        H_N=CA^{i-1}B
    \end{gathered}
    \label{eq:uncon_design}
\end{equation}
The expression of $z(k)$ is now substituted into the first part of the least square problem (containing the output).
\begin{equation}
    \phi_z=\frac{1}{2}\norm{\phi\,x_0+\Gamma\,U-R}_{Q_z}^2
\end{equation}
It is desired to have the form seen in \cref{eq:uncon_f}, why the above expression is rewritten by using linear algebra, which yields
\begin{equation}
    \phi_z=\frac{1}{2}U^T\underbrace{(\Gamma^TQ_z\Gamma)}_HU+{\underbrace{(-\Gamma^TQ_zK)}_{g}}^TU+\underbrace{\frac{1}{2}K^TQ_zK}_\rho\qquad , \qquad K=R-\phi x_0     
\end{equation}
Now the minimized control input variation (2\textsuperscript{nd} part) is analyzed.
\begin{equation}
    \phi_{\Delta_u}=\frac{1}{2}\,U^TH_sU+(M_{u1}\,u_{-1})^TU
\end{equation}
Where each element matrix is determined as
\begin{equation*}
    U=\begin{bmatrix}
    u_1\\u_2\\u_3\\\vdots\\u_N
    \end{bmatrix}\qquad
    H_s=\begin{bmatrix}
    2S & -S & 0 & \dots & 0\\
    -S & 2S & -S & \dots & 0\\
    0 & -S & 2S & \dots & 0\\
    \vdots & \vdots & \vdots & \ddots & \vdots\\
    0 & 0 & 0 & -S & S
    \end{bmatrix}\qquad
    M_{u1}=-\begin{bmatrix}
    S \\ 0 \\ 0 \\ \vdots \\ 0
    \end{bmatrix}
\end{equation*}
Now the two optimizations objects is added together, so the complete minimization is
\begin{equation}
    \label{eq:MPC_obj}
    \underset{U}{min}\,\phi=\phi_z+\phi_{\Delta u}=\frac{1}{2}U^THU+g^TU
\end{equation}
Where
\begin{equation}
    \label{eq:M_mat_MPC}
    \begin{gathered}
        H=H_z+H_s=\Gamma^TQ_z\Gamma+H_s\\
        g=M_{x0}\,x_0+M_R\,R+M_{u1}\,u_{-1}\\
        M_{x0} = \Gamma^TQ_z\phi \quad M_R=-\Gamma^TQ_z
    \end{gathered}
\end{equation}
\\\\
The implementation in \textit{MatLab} is carried out through three functions. Prior to the simulation, the MPC design i carried out (see \cref{app:MPC_design}) where, $H$, $H_s$ and $g$ is determined. In order to do this, the function calls a seperate function (see \cref{app:MPC_Constants}) where $\phi$ and $\Gamma$ is determined. Finally, the qpsolver is called. Notice, that since the simulation is unconstrained, the only inputs are $H$, $H_s$ and $g$ (see \cref{app:Uncon_MPC})